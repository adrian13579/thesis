\begin{opinion}
	La transformada wavelet se ha convertido en una de las técnicas más utilizadas para analizar las señales de audio e imágenes. Esta transformada se basa en funciones matemáticas especiales llamadas wavelets. La construcción de funciones wavelets es siempre un tema interesante y de gran aplicabilidad. Este es precisamente el tópico de esta tesis.

	En la literatura se reportan muchos enfoques de construcción que optimizan los parámetros matemáticos de dichas funciones como la regularidad, diferenciabilidad, momentos nulos, entre otras. Los métodos que permiten construir wavelets que sean capaces de reconocer en una señal determinados patrones definidos de antemano es menos frecuente. La creación de estas wavelets para patrones en 2 dimensiones (imágenes) no es tan frecuente en la literatura al respecto.

	La investigación realizada por el estudiante Adrian Rodriguez Portales se basa en la Transformada Shapelet Discreta II (DST-II) y propone tres alternativas de extensión 2D de dicha transformada. Para ello, primeramente, implementó la DST-II unidimensional y realizó experimentos para validar su eficiencia en la detección. Se compararon varios algoritmos para estimar la wavelet y se analizaron alternativas para su extensión a dos dimensiones. Se validó la propuesta en señales artificiales y en imágenes de mamografía digital. 

	Durante el desarrollo de la tesis Adrian estudió la literatura referente al análisis wavelet teórico de forma seria y crítica, proponiendo formas de cómputo de ciertas propiedades y parámetros involucrados en los algoritmos. Además, evaluó sus algoritmos en varias configuraciones experimentales (traslaciones, repeticiones del patrón, submuestreo del patrón) mostrando las ventajas y desventajas de este enfoque respecto a las wavelets clásicas. 

	Para esta tesis Adrian tuvo que estudiar las materias referidas, incluidas parcialmente en el currículo de la carrera, mostró disciplina, entrega y rigor. Además, demostró habilidades para el trabajo con la bibliografía y creatividad para proponer soluciones a problemas teórico-computacionales y de implementación, entre otras competencias de programación en el lenguaje Python y sus diversos frameworks. A pesar de que de las alternativas propuesta, solo uno mostró éxito en mamografía, considero que el estudiante logró cumplir exitosamente el objetivo propuesto y mostró que ciertos caminos no son adecuados para la extensión 2D de la DST-II, valor agregado de esta tesis.

	Por tanto, considero que a esta tesis del estudiante Adrian Rodriguez Portales debe otorgársele la máxima calificación (5 puntos, Excelente), y estoy seguro que en el futuro se desempeñará como un excelente profesional de la Ciencia de la Computación.

	\hspace*{\fill}\\
	\hspace*{\fill} MSc. Damian Valdés Santiago\\
    \hspace*{\fill} 26 de noviembre de 2022
\end{opinion}
