\begin{resumen}
	La transformada de wavelet se ha convertido en una herramienta sumamente útil en el campo de procesamiento
	de señales. Sin embargo, la gran variedad de wavelets existente hace que la búsqueda de la mejor para 
	un problema determinado sea un proceso bastante complejo. En los últimos años, se han desarrollado
	técnicas para construir wavelets
	adaptadas a los datos. Una de estas técnicas es la Transformada Shapelet Discreta , que
	permite la construcción de una wavelet (llamada \textit{shapelet}) capaz de detectar patrones a partir de los cuales
	se construyó. Este método presenta algunas ventajas con respecto a los algoritmos de aprendizaje automático 
	usados en el estado del 
	arte para la detección y reconocimiento de patrones, pues no es necesario un gran volumen de datos. Sin embargo,
	existen muy pocos trabajos sobre dicho algoritmo, y solo ha sido evaluado en señales unidimensionales, hasta 
	donde el autor conoce. En este
	trabajo se explora precisamente su extensión al caso de imágenes. Se proponen varias alternativas para su aplicación
	al caso señales de dos dimensiones y se evalúan los resultados en imágenes artificiales y en la detección de masas en mamografías.
\end{resumen}

\begin{abstract}
	The wavelet transform has become an extremely useful tool
	in the field of signal processing. However, the great variety of wavelets
	existing makes finding the best one for a given problem
	quite a complex process. In recent years, techniques to build 
	adapted wavelet to the data have been developed. One of these techniques is the 
	Discrete Shapelet Transform, which allows the construction of a wavelet (called \textit{shapelet})
	able to detect patterns from which it was built. This method presents
	some advantages with respect to machine learning algorithms used in the state-of-the-art 
	for the detection and recognition of patterns, since it is not necessary a large volume
	of data. However, there are very few works on this algorithm, and it has only
	been evaluated on one-dimensional signals, according to the reviewed literature. 
	This paper explores precisely
	its extension to the case of images. Some alternatives are proposed for their application to the case of
	two-dimensional signals and the results are evaluated in artificial images, and in the
	detection of masses in mammogramphies.
\end{abstract}
