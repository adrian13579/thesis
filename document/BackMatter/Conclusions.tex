\begin{conclusions}
	En este trabajo se replicó un algoritmo para la detección de patrones en señales unidimensionales
	y se exploraron vías para la extensión del mismo al caso bidimensional, explorando los resultados 
	de estas propuestas en señales artificiales y aplcándolo a la detección de masas en mamografías.
	
	La replicación del algoritmo logró encontrar un error en la definición de las ecuaciones de \textit{matching}
	en el artículo original, que al parecer debe ser un error de edición, porque de lo contrario no se hubieran
	podido lograr reproducir los resultados presentados en el artículo. Teniendo en cuenta esto último, se puede
	decir que se logró replicar el algoritmo exitosamente. Sin embargo, también se encontraron deficiencias en el
	mismo. Se pudo comprobar que el resultado de la DST-II en la detección de patrones depende en gran medida
	de la calidad de la solución numérica encontrada, si el error es alto no se logra una diferencia marcada
	entre el punto donde se encuentra el patrón y el resto. De los métodos probados, cuando el sistema contaba
	con más de 24 variables la gran mayoría de las veces no se alcanzaba convergencia, por lo que influía 
	directamente en la capacidad del algoritmo para detectar el patrón. Otra deficiencia encontrada del 
	algoritmo es su sensibilidad ante el ruido, es decir, solo es capaz de detectar al patrón exacto con el cúal 
	se diseñó la shapelet. 
	 
	Por último, la principal conclusión de esta tesis es que debido a las características del algoritmo (sobretodo las 
	condiciones de \textit{matching}) 
	no se puede extender al caso 2D, como sucede con otras wavelets. Ninguna de las propuestas mostradas aquí
	resultó ser eficaz, ni para señales artificiales ni para la detección de masas en mamografías.

\end{conclusions}
