\begin{conclusions}
	En este trabajo se replicó el algoritmo de la DST-II para la detección de patrones en 
	señales unidimensionales
	y se exploraron vías para la extensión del mismo al caso bidimensional, aplicando 
	estas propuestas en señales artificiales y en la detección de masas en mamografías.
	
	La replicación del algoritmo logró encontrar un error en la definición de las ecuaciones de detección 
	en el artículo original, provocando un cambio en dichas condiciones. Teniendo en cuenta esto último, se puede
	decir que se logró replicar el algoritmo exitosamente. 
	Se compararon distintos métodos de optimización para solucionar el sistema de ecuaciones no lineales, y
	se obtuvo que los más eficaces son el método de Levenberg-Marquardt y el método híbrido de Powell.

	También se encontraron deficiencias en el
	algoritmo de la DST-II. Se comprobó que el resultado de la transformada en la detección 
	de patrones depende en gran medida
	de la calidad de la solución numérica encontrada. Si el error es alto, no se logra una diferencia marcada
	entre el punto donde se encuentra el patrón y el resto. De los métodos probados, cuando el sistema contaba
	con más de 24 variables, la mayoría de las veces no se alcanzó la convergencia, lo que influyó 
	directamente en la capacidad del algoritmo para detectar el patrón. Otra deficiencia encontrada del 
	algoritmo es su sensibilidad ante el ruido, es decir, solo es capaz de detectar al patrón exacto con el cual 
	se diseñó la \textit{shapelet}. Esto es consecuencia de las condiciones de detección, que son muy estrictas.

	En el caso 2D, se exploró tres alternativas. Se evaluaron las mismas en señales artificiales que incluyen
	gaussianas, círculos y regiones rectangulares en distintas configuraciones. Se logró identificar
	las deficiencias de cada una de estas alternativas. También se usaron y evaluaron las tres
	propuestas en la detección de anomalías de tipo masa en mamografías.
	 
	Por último, la principal conclusión de esta tesis es que debido a las características del algoritmo (sobretodo las 
	condiciones de detección) 
	no se pudo extender exitosamente al caso 2D mediante las tres alternativas, como sí sucede con otras wavelets. 
	Ninguna de las propuestas mostradas aquí
	resultó ser eficaz, ni para señales artificiales ni para la detección de masas en mamografías.

\end{conclusions}
