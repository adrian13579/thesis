\chapter*{Introducción}\label{chapter:introduction}
\addcontentsline{toc}{chapter}{Introducción}

El cáncer de mama ocupa el lugar número 5 como causa de muerte en las mujeres a nivel
mundial. En 2020, un estimado de  684,996 mujeres alrededor del mundo murieron de
cáncer. Sin embargo, la detección temprana y posteriores tratamientos y 
han permitido disminuir
la tasa de mortalidad ha disminuido un 42\% desde 1989 a 2019\cite{breastcancer2020}
. De ahí que sea crucial la identificación y clasificación temprana de cualquier 
anomalía en el tejido mamario para el éxito en el tratamiento. Las mamografías de
rayos X son exámenes que brindan información esencial para este tipo de diagnóstico,
por lo que son consideradas entre las herramientas más efectivas 
para la detección de este tipo de enfermedades.

Las masas son un área de tejido mamario denso con una forma y bordes que hacen que se 
vea diferente al resto del tejido en esta zona y aunque en la mayoría de los casos
suelen ser benignas, pueden ser, en presencia de otras anomalías, un indicador 
de cáncer \cite{massbreast}.
En los últimos años se han utilizado técnicas del procesamiento de imágenes 
digitales para el perfeccionamiento de la detección de características propias
del cáncer y otros tipos de lesiones en las mamografías. 
Esto permite identificar patrones y anomalías que
sin previo procesamiento son muy difíciles de detectar para el ojo humano. 
Uno de los métodos más extendidos es la Transformada Discreta de Wavelet.

Desde que Alfréd Haar introdujo los conceptos fundamentales en 1909 \cite{haar} y años después
Ingrid Daubechies los consolidó \cite{daubechies1992}, la Transformada Discreta de Wavelet 
(DWT por sus siglas en inglés) ha sido la vanguardia en procesamiento de señales. Esta técnica 
permite fusionar información temporal y espectral al mismo tiempo. 

A lo largo de los
años se han desarrollado y utilizado distintos tipos de \textit{wavelets} para 
distintos problemas, algunas siendo mejores que otras para escenarios específicos. 
Para esto se han desarrollado numerosas técnicas que permiten obtener una base 
\textit{wavelet} a partir de las características del problema que se quiere
resolver. Este método se conoce como \textit{wavelet} 
adaptada (\textit{adapted wavelet}).

Entre las técnicas más recientes derivadas de la DWT para el procesamiento de
señales unidimensionales se encuentra 
la Transformada Discreta de \textit{Shapelet} 
(DST-I por sus siglas en inglés) presentada por primera vez en \cite{Guido2008} y
posteriormente mejorada con su segunda versión (DST-II) en \cite{Guido2018}.
De la misma forma que la DWT hace, la DST permite la localización 
en tiempo de la frecuencia, sin embargo con una ventaja especial:
también es capaz de medir le grado de similitud entre una forma predefinida
y la señal analizada. Esto ha permitido la aparición de una posible nueva herramienta para 
la detección de patrones en cualquier tipo de señales, incluidas imágenes.

En la Facultad de Matemática y Computación, se han realizado numerosas
investigaciones sobre el uso de 
varias técnicas en señales biomédicas. En particular, sobre el procesamiento
y detección de patrones en mamografías. El algoritmo propuesto en \cite{Guido2018} parece ser 
una alternativa que posee ciertas ventajas con otros enfoques de Machine Learning 
usados en los últimos años para este tipo de problemas y su validación y extensión para
el caso bidimensional podría ser una herramienta muy útil en el procesamiento de señales
biomédicas. Por este motivo y partiendo de la hipótesis de que la DST se 
puede extender para el caso bidimensional y aplicar 
para la detección de patrones en este tipo de
señales, este trabajo tiene como objetivo general el mejoramiento del contraste
en mamografías para la detección de masas. Para ello, se deben cumplir los 
siguientes objetivos específicos:

\begin{itemize}
	\item Revisión del estado del arte sobre la construcción de wavelets adaptadas 
	\item Evaluación numérica de algoritmos de optimización para estimar la \textit{wavelet}/(\textit{shapelet})
		usando la metodología propuesta en la DST. En el artículo \cite{Guido2018} no se muestra
		cuál algoritmo numérico es utilizado para resolver el sistema de ecuaciones no lineales, 
		por lo cual este es un punto importante en la presente investigación.
	\item Experimentación y validación de la DST en el caso de señales y patrones unidimensionales. Los resultados
	que presentados en el artículo \cite{Guido2018} carecen de cierto rigor y evidencia científica, por lo 
	que es necesario evaluar la eficacia del algoritmo.
	\item Implementación y extensión de la DST para el caso bidimensional. Este sería el principal aporte sobre
		la base del algoritmo propuesto en \cite{Guido2018}.
	\item Aplicación del algoritmo en mamografías para la detección de masas y evaluación de los resultados.  
	\item Implementación de un software (interfaz gráfica) para el uso del algoritmo propuesto.
\end{itemize}
