\chapter*{Introducción}\label{chapter:introduction}
\addcontentsline{toc}{chapter}{Introducción}

Muchos procesos de la naturaleza se pueden modelar como señales. El sonido, la luz y la actividad
cerebral de una persona son todos fenómenos que se representan a través de señales. Cada una con
sus características únicas. Esto ha motivado el estudio y la creación de herramientas capaces
de descubrir los secretos de las mismas. 

En 1822, Joseph Fourier publica su \textit{ Théorie analytique de la chaleur } \cite{Fourier2009} donde introduce la idea de usar
series matemáticas para analizar la conducción del calor en cuerpos sólidos. Esta teoría se conoce en
la actualidad como Teoría de Fourier. En la misma, se plantea que una señal se puede representar como
una serie, posiblemente infinita, de senos y cosenos. Este enfoque permite estudiar las diferentes 
frecuencias presentes en una señal y la amplitud de las ondas que la conforman. A la herramienta que permite
analizar el comportamiento de una función en el dominio de la frecuencia se le conoce en la actualidad como
Transformada de Fourier.

Las propiedades de esta transformada, la convierten en una herramienta muy poderosa para muchos escenarios. 
Sin embargo, su principal desventaja es que cuando se pasa al dominio de las frecuencias, la información 
temporal se pierde. La información de la frecuencia es fija en relación con el tiempo en una función 
trigonométrica. Esto es el caso contrario al de muchas señales, como el sonido, donde las características
cambian a lo largo del tiempo. 

En 1910 , Alfréd Haar introduce en su trabajo \textit{Zur Theorie der orthogonalen Funktionensysteme} \cite{Haar1910} un tipo 
funciones que en la actualidad se conocen como wavelets, pero no es
después de varias décadas que se les acuña este nombre. Estas funciones son como pequeñas ondas y a diferencia
de las funciones trigonométricas, son de pequeña duración y localizadas en tiempo permitiendo obtener información tanto temporal como
de frecuencia. Otra ventaja con respecto a la bases de Fourier, es que las bases wavelets no son únicas: 
existen una amplia variedad, para todo tipo de aplicaciones. 

La propagación de las wavelets en la comunidad científica y académica es sorprendente.
El surgimiento y consolidación de esta teoría a mediados de los 1980s, su conexión con el procesamiento
de señales y bancos de filtros \cite{Mallat2008}, el diseño de una algoritmo eficiente para el cómputo de la transformada \cite{Mallat2008}
y los aportes sobre el estudio de wavelet ortogonales \cite{daubechies1992} son algunos hechos que marcan su rápido desarrollo en
pocos años.   

Dejando un lado el aspecto puramente matemático, las wavelets también han tenido un gran impacto  
en muchos campos y problemas. Por ejemplo, el algoritmo de compresión de imágenes JPEG 2000 está basado en
bases wavelets \cite{Taubman2002}.

En el campo de la medicina las wavelets también han hecho una incursión bastante importante. Trabajos como
\cite{Bhattacharyya2018} y \cite{Sharma2020} muestran su aplicación en la detección de patrones y áreas de interés
en electroencefalogramas (EEG) respectivamente. Estudios como \cite{Too2018} han hecho una comparación de las distintos
tipos de wavelets en los EEG. 

En el procesamiento de imágenes médicas también han sido 
usadas en los útimos años, por ejemplo, en la compresión \cite{Bruylants2015}\cite{Alkinani2021}, en la eliminación de ruido
 \cite{Wang2006}\cite{George2016}\cite{Patil2021} y para 
mejorar el contraste \cite{Dikshit2022}.

Las wavelets también han evolucionado y han sido la base para nuevos algoritmos que añaden nuevas capacidades.
Tal es el caso de transformada discreta de \textit{shapelet} (DST) \cite{Guido2008} que está basada en la transformada discreta
de wavelet (DWT), pero que añade información de forma en conjunto con el clásico análisis temporal y de frecuencia de la
DWT. En un segundo trabajo se crea una segunda versión de este algoritmo, DST-II \cite{Guido2018}, y en \cite{Guido2021} se hacen 
algunas modificacones para obtener simetría, creando una tercera versión conocida como DST-III.
La DST, aunque poco estudiada, por sus propiedades de detección puede se una alternativa a otros algoritmos de
detección de patrones usados en el estado del arte que están basados en \textit{machine learning} y que requieren
de muchos datos para us correcto funcionamiento.

En la Facultad de Matemática y Computación, también se trabaja con las wavelets y su transformada, sobretodo en
investigaciones sobre el procesamiento de imágenes biomédicas. Las mamografías son uno de los tipos de imágenes
con las que se trabaja en este centro. 
Con la aparición del algoritmo de la DST-II, surge  
una alternativa que posee ciertas ventajas con otros enfoques  
usados en los últimos años para problemas de detección. Sin embargo, su uso solo ha sido estudiado para el 
caso de señales unidimensionales, por lo que es de interés para el centro experimentar y explorar su extensión
para el caso de señales bidimensionales. Por este motivo y partiendo de la hipótesis de que la DST se 
puede extender para el caso bidimensional y aplicar para la detección de patrones en mamografías, este trabajo tiene como 
objetivo general explorar el uso de la DST-II para detectar masas en mamografías. Para ello, se deben cumplir los 
siguientes objetivos específicos:

\begin{itemize}
	\item Revisar el estado del arte sobre la construcción de wavelets adaptadas 
	\item Evaluar algoritmos de optimización para estimar la \textit{wavelet}(\textit{shapelet})
		usando la metodología propuesta en la DST-II. En el artículo \cite{Guido2018} no se muestra
		cuál algoritmo numérico es utilizado para resolver el sistema de ecuaciones no lineales, 
		por lo cual este es un punto importante en la presente investigación.
	\item Replicar la DST-II en el caso de señales y patrones unidimensionales. 
	\item Explorar ideas para la extensión de la DST-II para el caso bidimensional. 
	\item Aplicar las ideas para la extensión de al DST-II en mamografías para la detección de masas de los resultados.  
	\item Implementar un software (interfaz gráfica) para el uso del algoritmo y las propuestas para su extensión al caso de señales
		bidimensionales.
\end{itemize}

El presente documento está organizado en 3 capítulos.

En el capítulo 1 se introducen los conceptos fundamentales realacionados con la transformada de wavelet. Se definen
la transformada continua de wavelet, la transformada discreta de wavelet y los bancos de filtros.
También se presenta el algoritmo de Mallat y se mencionan algunas de las familias de wavelets más conocidas.
Por último, se mencionan algunas de las técnicas usadas en la construcción de wavelets adaptadas a los datos. 

En el capítulo 2 se expone la propuesta de este trabajo para construcción de wavelet adaptadas a patrones: 
la DST-II. Se hace una descripción detallada del algoritmo y se exponen algunas ideas para su extensión en el
caso de señales bidimensionales.

En el capítulo 3 se muestran los resultados de la replicación de este algoritmo a través de varios experimentos.
Se hace una pequeña comparación entre distintos métodos numéricos y las soluciones de los mismos al sistema de ecuaciones
no lineales. Se muestran también los resultados de los experimentos para verificar la eficacia del algoritmo implementado en  
la detección de patrones en señales de una dimensión. Luego, se exponen  los resultados y el análisis de las propuestas
para su extensión al caso de señales de dos dimensiones, y su aplicación en la detección de masas en mamografías.

